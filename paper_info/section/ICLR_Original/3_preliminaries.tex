\section{Preliminaries}
\subsection{Human Gaussian Splatting}

Recent advances have rapidly progressed in applying 3D Gaussian Splatting (3DGS)~\cite{kerbl20233d} techniques to human body modeling.
 In particular, one line of work defines a canonical space based on an SMPL mesh~\cite{loper2023smpl} and uses multiple MLP networks to predict Gaussian attributes (geometry, color, and scale) variations. 
The canonical Gaussian representation is then deformed to match a target pose for each frame using LBS~\cite{loper2023smpl}. 
In other words, the method can be formulated as: 

\begin{equation}
\psi_c(f(\mathbf{T}_c)) = (\mathbf{c}, \mathbf{o}, \Delta P_c, \mathbf{R}, \mathbf{S}, \mathbf{W}), \tag{1}
\end{equation} 
\begin{equation}
P_{def} = LBS(P_c,; \mathbf{\theta},; \mathbf{W}) , \tag{2}
\label{eq_1}
\end{equation}

where $\theta$ denotes the set of SMPL pose parameters. Equation~(1) represents the extraction of Gaussian properties (color $\mathbf{c}$, opacity $\mathbf{o}$, position offset $\Delta P_c$, rotation $\mathbf{R}$, scale $\mathbf{S}$, and skinning weights $\mathbf{W}$) from the canonical space, 
while Equation~(2) applies the LBS function to obtain the deformed positions $P_{def}$ of the Gaussians in the posed space. More recently, research \cite{moon2024expressive} has included triplane inputs to MLPs, taking various parameters as input, 
to reproduce more realistic movements. This method uses multiple MLPs, taking triplane features and 
3D joint poses (excluding the root pose) as inputs, to predict $A_{\text{tri}}=\{\Delta P_{\text{tri}},\,S_{\text{tri}},\,C_{\text{tri}}\}$ and $A_{\text{pose}}=\{\Delta P_{\text{pose}},\,\Delta S_{\text{pose}},\,\Delta C_{\text{pose}}\}$. 
Additionally, $A_{\text{pose}}$ takes the normal vector of each vertex as an extra input for prediction, thereby reflecting the fine deformations that occur during animation. 